\begin{center}
\large{\textbf {RESUMEN}}
\end{center}

Este estudio proporciona por primera vez un análisis de la disponibilidad de los recursos hidricos a escala de cuenca hidrográfica en el Perú. Utilizando nuevos conjuntos de datos grillados de precipitación y temperatura, junto con seis estimaciones reales de evapotranspiración de productos de percepción remota, se evalúo la vulnerabilidad de los recursos hídricos debido al cambio climático. Esto se aborda bajo un enfoque ascendente y un marco probabilístico de Budyko. Primero, se realiza una clasificación de los productos de percepción remota bajo en enfoque de balance hídrico. Luego, se aplica el Budyko probabilístico y se valida de forma cruzada utilizando la inferencia de evapotranspiración real adecuada. Finalmente, la vulnerabilidad del agua y la incertidumbre asociada se calculan a partir del Budyko probabilístico calibrado junto con los espacios climáticos a partir de las variaciones de evapotranspiración potencial (de temperatura) y precipitación. Los principales resultados mostraron que los mejores productos para usar en Perú fueron: GLEAM, MEDIAN, TerraClimate, ZHANG, MODIS16 y SSEBop. Además, el probabilístico Budyko ofreció un buen rendimiento, principalmente en la vertiente hidrográfica del Amazonas. Adicionalmente, las cuencas ubicadas en los Andes, especialmente en el sur, mostraron un menor cambio crítico de precipitación (menos del 10 \%) para aumentar la vulnerabilidad de la disponibilidad de los recursos hídricos en un 25\%.

\textbf {Palabras clave:} Budyko probabilístico, evapotranspiración actual, bottom-up, cambio climático.

\clearpage

\begin{center}
\large{\textbf {ABSTRACT}}
\end{center}

This study provides for the-first-time a water availability analysis at basin-scale in Peru. Using new gridded datasets of precipitation and temperature, along with six actual evapotranspiration estimations from remote sensing products, the vulnerability of water resources due to climate change is evaluated. This is addressed under a bottom-up approach and probabilistic Budyko framework. First, a ranking of remote sensing products is done under a water-balance view. Later, the probabilistic Budyko is applied and cross-validated using the adequated actual evapotranspiration inference. Finally, water vulnerability and associated uncertainty are computed from the calibrated probabilistic Budyko along with climate spaces from variations of potential evapotranspiration (from temperature) and precipitation. The main results showed that the best products to use in Peru were: GLEAM, MEDIAN, TerraClimate, Zhang, MODIS16, and SSEBop. In addition, the probabilistic Budyko provides good performance, mainly in the Amazon River watershed. Furthermore, basins located in the Andes, especially in the southern, showed lower critical precipitation change (less than 10\%) to increase the vulnerability of water availability in 25\%.

\textbf {Key words:} probabilistic Budyko, actual evapotranspiration, bottom-up, climate change.

\clearpage