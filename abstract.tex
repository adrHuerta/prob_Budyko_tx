\begin{center}
\large{\textbf {RESUMEN}}
\end{center}

Este estudio proporciona un análisis de la disponibilidad de recursos hídricos a escala de vertiente y cuenca en Perú. Utilizando nuevos datos grillados de precipitación y temperatura, junto con seis estimaciones de evapotranspiración real de productos de sensoramiento remoto, se determinó la vulnerabilidad de los recursos hídricos debido al cambio climático. Esto se aborda bajo un enfoque abajo-arriba y el marco probabilístico de Budyko que permite medir la incertidumbre asociada. Primero, para seleccionar una estimación adecuada de la evapotranspiración real a largo plazo, comparamos a escala de cuenca los productos con la evapotranspiración real inferida por balance hídrico y Budyko deterministico. Luego, el Budyko probabilístico se calibra utilizando la evapotranspiración real adecuada y se valida de forma cruzada a escala de país, vertiente y cuenca. Finalmente, la vulnerabilidad de disponibilidad de los recursos hídricos y la incertidumbre asociada se calcula junto con los espacios climáticos hipotéticos a partir de las variaciones de evapotranspiración potencial y precipitación. Los resultados principales muestran que TerraClimate, GLEAM y PROMEDIO son los productos mejor calificados en términos de bajo sesgo, RMSE y alto R. TerraClimate y PROMEDIO presentan sesgo y RMSE más bajos, y TerraClimate estima muy bien la distribución espacial de la evapotranspiración real (R mejor clasificada). Por el contrario, MODIS16, SSEBop y P‐LSH son menos eficientes. Por lo tanto, como referencia para la evapotranspiración real, seleccionamos PROMEDIO que representa el promedio lineal de los productos. Al lograr las tres variables, calibramos y validamos de forma cruzada el Budyko probabilístico en términos del índice de evaporación. La evidencia sugiere que la distribución regional del parámetro Budyko alcanza errores de $\pm$2\% a escala de país y de vertiente, y $\pm$9\% como promedio a escala de cuenca. Por lo tanto, el marco propuesto proporciona un gran rendimiento. Con base en esta evaluación, descubrimos que las cuencas ubicadas en los Andes, especialmente en el centro y sur, mostraron un cambio crítico de precipitación más bajo (menos del 10\%) para aumentar la vulnerabilidad de la disponibilidad de recursos hídricos en un 25\%.

\textbf {Palabras clave:} Budyko probabilístico, evapotranspiración actual, abajo-arriba, cambio climático, recursos hídricos.

\clearpage

\begin{center}
\large{\textbf {ABSTRACT}}
\end{center}

This study provides for the-first-time a water availability analysis at drainage and basin-scale in Peru. Using new gridded datasets of precipitation and temperature, along with six global actual evapotranspiration estimations from remote sensing products, the vulnerability of water resources due to climate change is assessed. This is addressed under a bottom-up approach and probabilistic Budyko framework that enables us to measure the associated uncertainty. First, to select an adequate estimation of long-term actual evapotranspiration, I compared at basin-scale the remote sensing products with long-term actual evapotranspiration inferred from a water-balance and deterministic Budyko. Later, the probabilistic Budyko is calibrated using the adequated remote-sensed actual evapotranspiration and is cross-validated at country, drainage, and basin-scale. Finally, the water availability vulnerability and associated uncertainty is computed along with climate spaces from variations of potential evapotranspiration (from temperature) and precipitation. The main results show that TerraClimate, GLEAM and PROMEDIO are the highest-ranked products in terms of estimation of long-term mean actual evapotranspiration across basins with low bias, RMSE, and high R. TerraClimate and PROMEDIO present lower bias and RMSE, and TerraClimate estimate very well the spatial distribution of actual evapotranspiration (highest-ranked R). On the contrary, MODIS16, SSEBop and P‐LSH are less efficient based on most criteria evaluation. Therefore, as a reference for actual evapotranspiration, is selected PROMEDIO which represents the linear averaging of remotely sensed products. From this perspective, it is calibrated and cross-validated the probabilistic Budyko in terms of the evaporative index. The evidence suggests that the regional distribution of the Budyko parameter accomplishes errors of $\pm$2\% at the country and drainage-scale and $\pm$9\% as average at basin-scale. Thus, the proposed framework provides great performance. Based on this evaluation, we figure out that basins located in the Andes, especially in the southern, showed lower critical precipitation change (less than 10\%) to increase the vulnerability of water availability by 25\%.

\textbf {Key words:} probabilistic Budyko, actual evapotranspiration, bottom-up, climate change.

\clearpage