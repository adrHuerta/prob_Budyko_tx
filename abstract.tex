\begin{center}
\large{\textbf {RESUMEN}}
\end{center}

This study provide a regional assessment of trends in extreme indices of precipitation in the department of Puno. It used a database of 19 stations with records of daily precipitation (period: 1971 - 2013). It computed eleven extreme indices of precipitation which characterize the precipitation regime, analizing in first place change points and subsequently trends. It performed multiple statistical tests across the implementation of false discovery rate procedure focusing to evaluate regional significance test. The spatial variability as well as the interanual variability has been evaluated and related with different oceanic - atmospheric indices as, El Niño–Southern Oscillation (ENSO) and the sea surface temperature of Atlantic Ocean. The results showed that there is not a strong trend toward wet or dry conditions, only in the southern part of the department was found decreases in length of wet days. Furthemore, almost or very little significant trends were found in extreme indices related to heavy precipitation events. Indices linked to ENSO are negative correlated to total precipitation, and in a strong way to the length of wet days; while that indices of heavy precipitation show a strong spatial variability and they are weakly correlated to oceanic - atmospheric indices.

\textbf {Palabras clave:} Budyko probabilístico, evapotranspiración actual, bottom-up, cambio climático.

\clearpage

\begin{center}
\large{\textbf {ABSTRACT}}
\end{center}

This study provide a regional assessment of trends in extreme indices of precipitation in the department of Puno. It used a database of 19 stations with records of daily precipitation (period: 1971 - 2013). It computed eleven extreme indices of precipitation which characterize the precipitation regime, analizing in first place change points and subsequently trends. It performed multiple statistical tests across the implementation of false discovery rate procedure focusing to evaluate regional significance test. The spatial variability as well as the interanual variability has been evaluated and related with different oceanic - atmospheric indices as, El Niño–Southern Oscillation (ENSO) and the sea surface temperature of Atlantic Ocean. The results showed that there is not a strong trend toward wet or dry conditions, only in the southern part of the department was found decreases in length of wet days. Furthemore, almost or very little significant trends were found in extreme indices related to heavy precipitation events. Indices linked to ENSO are negative correlated to total precipitation, and in a strong way to the length of wet days; while that indices of heavy precipitation show a strong spatial variability and they are weakly correlated to oceanic - atmospheric indices.

\textbf {Key words:} probabilistic Budyko, actual evapotranspiration, bottom-up, climate change.

\clearpage