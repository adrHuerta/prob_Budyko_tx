% Please add the following required packages to your document preamble:
% \usepackage{longtable}
% Note: It may be necessary to compile the document several times to get a multi-page table to line up properly
\begin{longtable}{l|l|l}
\caption{Resumen de los principales métodos de estimación de evapotranspiración actual basado en percepción remota. \label{tab:TableZhang02}} \\
\hline
\textbf{Método} & \textbf{Ventajas}        & \textbf{Asunciones y/o}             \\
\endfirsthead
%
\multicolumn{3}{c}%
{{\textbf{$<<$continuación$>>$} \vspace{.5cm}}} \\
\endhead
%
\hline
\endfoot
%
\endlastfoot
%
\textbf{}       & \textbf{}                & \textbf{limitaciones}               \\ \hline
SEB             &                          & Solo para cielo despejado;          \\
Una fuente      & Simple, y poco           & requiere excesiva                   \\
                & requerimiento de         & parametrización de                  \\ \cline{1-1}
SEB             & información              & resistencia y calibración local;    \\
Una fuente con  & meteorológica            & probable a errores de $LST$ y $T$;      \\
variabilidad    &                          & requiere valores            \\
espacial        &                          & instantáneos a diarios              \\ \hline
SEB             &                          &                                     \\
Dos fuentes     & Poco requerimiento       & Solo para cielo despejado;          \\ \cline{1-1}
SEB             & de información           & muy sensible a los errores de $LST$;   \\
Dos fuentes     &  meteorológica           & requiere valores          \\
(tiempo         &                          & instantáneos a diarios              \\
diferenciado)   &                          &                                     \\ \hline
LST-VI          & Poco requerimiento      & Solo  para cielo despejado;         \\
                & de información           & Relación LST-VI muy simplificado;   \\
                & bajo impacto en          &   requiere valores                  \\
                & errores de $LST$         & instantáneos a diario     \\ \hline
PM              & Cobertura continua       & Alto requerimiento de datos;          \\
                & basada en procesos       & Estimación simplificada o           \\
                & físicos.                  & semi-empírica de la                 \\
                & Paso de tiempo flexible, & conductancia del dosel              \\
                & sin requisitos o con     &                                     \\
                & requisitos bajos de $LST$  &                                     \\ \hline
PT              & Simple; requerimiento    & Muchas simplificaciones físicas;     \\
                & moderado de              & requiere flujo de calor del suelo   \\
                & información              & como entrada o supone que           \\
                &                          & es despreciable; aplicado             \\ 
                &                          & a escala mensual                    \\ \hline
MEP             & Poca información         & Requiere $LST$ continuo para            \\
                & meteorológica            & producir un registro $AE$ continuo.     \\ \hline
Balance         & Simple y fácil           & No puede derivar directamente $AE$;    \\
Hídrico         & de aplicar               & pobre resolución espacio-temporal   \\
                &                          & sensible a los errores de $P$         \\ \hline
Agua -          & Considerando relación    & Posible requerimiento de mucha      \\
Carbono         & entre el carbono y       & información, afectado por           \\
                & flujos de agua           & vacíos y/o errores.                  \\ \hline
Modelos         & simple y fácil           & Requiere calibración y capacidad     \\
Empíricos       & de aplicar               & baja fuera del área de calibración; \\
                &                          & proceso físico simplificado;      \\
                &                          & sujeto a condiciones climáticas     \\
                &                          & si se requiere $LST$                   \\ \hline
\end{longtable}
\vspace*{-1.25cm}
FUENTE: \citet{zhang2016review}

